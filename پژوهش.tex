\documentclass{book}
\usepackage{amsmath,amsthm,amssymb,mathrsfs,amscd,mathtools,dsfont,nccmath}
\usepackage{xepersian}
\settextfont[]{Arial}
\begin{document}
\chapter{NET-BASED DISSEMINATION OF e-RESEARCH RESULTS}
The dissemination phase of e-research is the climax of the research cycle, and it occurs
when research share the results of their important research studeies with the world.
Unfortunately, it is this stage that is fraught with indecision and unhealthy bouts of
procrasting. in some cases,the funders of the e-researchset deadlines that serve to 
motivate and pace the researcher through this final sprint. However, all-too-often in
academic research, the dissemination phase comes afterthe defense of a formal thesis
or major project,when the grant money has been spent, and in both case, energy may
be low. the result is that too often the outcomes of significant research studies remain
hidden in a bound theise resting on a dusty library shelf or filed as a completed research
report in a bureaucrats office.
We begin this chapter with an overview of the reasons why the dissemination
process is important and worth putting forth the extra energe that is involved. We then 
provide an overview on the selection of the networked tools best suited for this task
and end with tips for effectively and efficiently disseminating the finding of a research project.
worth saying and your research efforts will make a contributon (however major or minor)
to improving education and opportunities for learning. there are also more 
mundane and practial reasons for disseminating the results of authors sponsored by 
MCB University Press, provides a listing of four compolling reasons to publish.\\

Beacuse I Have To\\

begin a professor (or undertaking any of the roles associated in the long apprentieeship
tail from graduate student through assistant and associate professor) means that one
is obliged to profess.Webster's Reuised UNabridges Dictionary (1998) defines perofess as "to
make  open deelaration of, as of one's knowledge, belief, action."Thus, you are asked,
as a member of the  research community, to publicy declare what you know and what
your research has revealed. This declaration is a form of accountability in which you 
show that the time and effort spent on your study is justified and worthy of your per-
sonal as well as societal support.
Disseminating your materials also repays your debt to those whose ideas and
efforts have assisted the work. the research process that you are now completing has
been aided by many other professors  who have publicly given their insights into related
problem, methodologies, and solutions. publishing also helps repay those participants
who have given their time and insights to you throughout the research process. it is 
mow your turn to repay this social debt and add to the accumulated public knowledge
and wisdom. this is not only a great responsibility, but it is also a great honor. the dis-
semination process, like all components of the  e-research cycle , is marked by hard
work and attention to detail. however , it is also nearest to the pay-off stage and thus
can be the most rewarding component of the e-research process.\\

Beacuse I Want to Get Ahead\\

Public dissemination of resultes often resultes in a flurry of contacts and connections with
other researchers.Gaining a reoutation as a competent research who creates  valuable
knowledge and who knows how to communicate these results provides a focus to which 
additional opportunities gravitate. these may include offers to collaborate on future
work, invitations to speak at conferenees or other gatherings,, invitations to travel and
visit with other researchers, requests for advice or offers of further research or related
employment. in academic circles, publishings, invitations to travel and
visit eith other researchers , requests for advise or offers of further research or related
employment. in academic circles, publishing, especially in peer-reviewed journal articles
and books , is one  of the few quantifiable contributions to schplarly life and yhus the
count of publications often takes inordinate importance in promotion, tenure, and 
salary decisions. Graduate students who have the beginnings of a publishing record on 
their resumes are inevitably more sought after than those who can show competence
only through successful completion of courses. there are a variety of personal and 
institutional ego-related reasons for disseminating as well. in sum there are many com-
pelling reasons why dissemination of results is a very reliable indicator of future success.\\


Beacuse I Need Learn from Others\\

the public nature of dissemination means that others will critically read and reflect on 
your work. this review may be highly formalized and undertaken by peers whose
identities are  hidden from you.Alternativly, the review may take the from of a thorough
proofreading by a skilled editor. ULtimately, countless members of the general
public review your results. Each of these  reviewers will have suggestions, concers, and
even major problem with your research. Rarther thsn viewing this feedback as deterrents 
to dissemination, these reviews provide opportunity to hone the results and present them
in ways that clearly and succinctly reval the knowledge that you have created.
good research is a dialogue between the researcher and the many potential
consumers of the results. this dialogue often continues with ideas you choose to pursue,
identification of new insights, and applications of the results--as well ass opportunities 
for building on your prior research bu identifying new qusetions and new  opportunities to 
pursue these questions.\\

Beacuse I Need to Clarify My Own Thoughts\\

the final reason provided by the literati club is to meet the specific needs of many
different audiences.E-researchers, as authors , need to focus on particular audiences as 
they present the results of their work. By asking themeselves what these results mean to 
different groups of knowledge consumers, additional clarity and insight often arises.
this insight is then honed through successive rewrites and precention formats until
the research is packaged in such a way that the results are perceived ad both accessible 
and of value to the imtended auduence(s).normally this revision and focusing exercise brings
additinal insights into the work that enhance its usefulence, not only to the audience but
even more so to the researcher.

Beacuse It Can Be Satisfying\\
We add a fifth reason to publish , which may not always be attested to by struggling
e-researcher. The dissemination stage is one of expencion. the development and teseing 
of ideas is often a shrinking and focusing activity an which meaning as created by
tightly focusing research tools on carefully defined and cricumscribed events, ideas, or
activities. in the dissemination phase , researchers are able to expand thier thinking by
creating and sharing not only the direct results but also  the amp;ications for practice
and for further research, this opprtunity to reveal the significance of your research is 
often the most satisfying component of the research process.\\
the dissemination proccess can provide entertaining , creative, and interactive
learning experiences when you publish your results using the multimedia capacity of the Net.
the development of engaging Web sites that use Webs multimedia capacities can facilitate new learning 
as one masters graphics, Web creation, discussion forums, and other interactive tools.
this learning and playimg with the presentation of  your results is usually enjoyable work and can
be a most satisfying component of the  e-research process.\\


CREATING QUALITY CONTENT\\

tere are many guidws to academic writing that describe the process and format of disseminating 
your fimdings. rather than focus on these more generic skills, we look at the dicerse ways in which writing for the Net is unique
and the  multitude of ways iin  which the Net can be used to diseminate the resultes of your work.
research has shown that we process information from the screen in ways that are different from the way we read texts 
or paper content (Kanuka & Szaboo, 1999). Net readers are more likely to skim rather than read meticulously through screen presented content.
Thus, e-researchers should use  the techniques of the newspaper editor, rather than the novelist to present their finding an a Web document.
for example, the style of screen formatted materials should make extensive use of white space.
these formatting techniques allow the reader to focuse on items of particular interest and skim through that which is not of interest.
for these same resons , the content should be concise and to the point.
some expert suggest using the inverted pyramid style of presentation that was developed by newspaper editors and reportes.
Unlike traditinal research papers that begin with an unresolved problem, then present
all the  relevant past research and methodology, and finally conclude with resultes and applications, an  inverted
pyramid style begins with the most important and relevant content. less relevant and more detailed content is placed at, or near,
the end of the article. includentally, this  style  not only allows busy readers the capacity to stop reading at any time knowing they 
have already coverded the most relevant material, but it also provides the writer or editor the capacity to omit
content from the end of the article when space problem arise.
while the space element does not apply to the wev (with the exception of server space), the psychological benefits of 
brevity remain as relevant as ever.\\
Despite the large number of prescriptive  guidelines and articles for web writing 
(for example see introduction to hypertext writing style at http://www.bu.edu/cdaly/hyper.html),we are also  aware that the nature of the 
web,and readers appproach to screen reading, is changing. New technologies (includeing electronic paper and very high resolution screens)
as well as evidence of successful Net publication using a variety of  writing formats and styles, remind us that the Net thrives on diversity and that 
there is no single formula for all forms of effective research results dissemination(Bresler,2000).\\
the e-researcher's goal is to work the content into a form that is clearly and easily understood by the intended audience. this will include
detailing the study and for whom the resultes will b of interest.it should also provide enough.\\
\end{document}
