\documentclass{book}
\usepackage{amsmath,amsthm,amssymb,mathrsfs,amscd,mathtools,dsfont,nccmath}
\begin{document}
4.Consider the following Turning machine\\
\begin{align}
\delta(A,a)\rightarrow(B, a,R\rightarrow)\\
\delta(B,b)\rightarrow(B, b,R\rightarrow)\\
\delta(B,a)\rightarrow(B, a,R\rightarrow)\\
\delta(C,b)\rightarrow(B, b,R\rightarrow)\\
\delta(C,a)\rightarrow(B, a,R\rightarrow)\\
\end{align}

d is the state.the language accepted by the TM is\\
\begin{align} 
a)  ab\ast ab \ast      b) ab\ast ab\ast a        c) ab + ab +a		d) a\ast(ba)\ast a\\
\end{align}
 
5.Consider the following Turing machine \\
\begin{align}
\delta(A,a)\rightarrow(B, a,R\rightarrow)\\
\delta(A,b)\rightarrow(B, b,R\rightarrow)\\
\delta(B,a)\rightarrow(B, a,L\rightarrow)\\
\delta(B,b)\rightarrow(B, b,L\rightarrow)\\
 
 \delta(B,Blank)\rightarrow(C, Blank,R\rightarrow)\\
 \end{align}
C  is the final state\\
On input ab , the machine will\\
a) Halt on acceoting state				b) Go into infinite loop\\
c) Crash										d) Reach to final state but willl not halt\\

6.When does a Turing machine crash?\\

a) if the machine traverses all the inputs without  traversing some state\\
b) if it traverses all its states till the input remains\\
c) if the transitional function is not  defined for the present state and the input combimation\\
d) None of this\\

7. A single-tap Turing machine M has two states q_{0} and q_{1} , of which  q_{0} is the starting state. the tape alphabet
of M is \lbrace 0,1,B\rbrace and its input alphabet is \lbrace 0,1\rbrace . the symbol B is the blank symbol used to indicate the end
of an input string. the transition function of M is described in the following table.\\

The table is interpreted as illustrated in the following.\\
the entry (q_{1},1,R) in row q_{0} and coulmn 1 signifies that if M is in state q_{0} and reads 1 on the curent tape square, then it writes 1 on the same tape square, moves its tape head one position to the right , and transition to state q_{1}.\\
which of the following  statements is true about M?\\
\begin{align}
a) M does not halt on any string in (0 +1 )^{+}\\
b)  M does not halt on any string in (00 +1 )^{\ast}\\
c)	 M halt on all strings ending in a 0\\
d)	M halt on all strings ending in a 1\\
\end{align}
8.Consider the language L_{1},L_{2}, and L_{3} as given in the following\\
\begin{align}
L_{1} = \lbrace 0^{p} 1^{q} \mid p,q \in \rbrace \\
L_{1} = \lbrace 0^{p} 1^{q} \mid p,q \in N  and p=q\rbrace \\
L_{1} = \lbrace 0^{p} 1^{qr} \mid p,q ,r\in  N and p=q=r\rbrace \\
\end{align}
 which of the following statement is not true?\\
 a) Pushdown automata (PDA) can be used to recognize L_{1} AND L_{2}\\ 
 b) L_{1} is a regular language\\
 c) All the three language are context free\\
 d) Turing machines can be used to recognize all the language\\

Answers:\\
1. d			2. c			3. c			4. b 			5. b 				6. c			7. a			8. c\\

Hints:\\
\begin{align}
5. Aab\rightarrow aBb \rightarrow Aab \longrightarrow aBb....infinite times\\
7. For (b),(c), and (d) the machine loops infinitely. it only accepts null string.\\
8. L_{3} Is not context free.\\
\end{align}

\chapter{Fill in the Blanks}\\
\begin{align}

1. All types of language are accepted by\cdots\cdots\cdots\\
2. According to the chomsky hierarchy, type 0 language is called\cdots\cdots\cdots\\
3. The diagram of the Turing machine is like finite automata, but here the head moves\cdots\cdots\cdots\\
4. The head of the turing machine is called\cdots\cdots\cdots\\
5. The string a^{n}b^{n}c^{n}, n>0 is accepted by \cdots\cdots\cdots\\
6. The turing machine is called\cdots\cdots\cdots\\
7. Two-stack PDA is equivalent to\cdots\cdots\cdots\\
8. The crash condition occurse if the read-write head of a Turing machine is over the\cdots\cdots\cdots\\
9. A turing machine is said to be in \cdots\cdots\cdots if it not able to move future.\\

Answers:\\

1. Turing machine			2. 	unrestricted language			3. in both direction\\
4. Read-write head		5. Turing machine						6. universal machine\\
7. Turing machine			8. leftmost cell							9. halt state\\
\end{align}

\chapter{Exercise}\\

1. Construct Turing machine for the following\\
\begin{align}
a) L = \lbrace a,b\rbrace ^{+}\\
b) L= WW, Where W \in (a,b)^{+}\\
c) L= WW^{R}, Where W \in (a,b)^{+}\\
d) L= \lbrace all even palindromes over (a,b)\rbrace\\
e) L= a^{n}b^{n}c^{n}, n>0\\
f) L= n-1, where n>0\\
\end{align}
\begin{align}
2. Design a Turing machine which acts as an eraser.\\
(Hints: Starts from the left hand side, scans each symbol from left to night, and  replaces each symbol by blank. halts if gets blank.)\\

3.Design a Turing machine which replace 0 by 1 and 1 by 0 of the string traversed.\\

4. Design a Turing machine to accept the string L= \lbrace a^{n}b^{m}c^{n+m}, where n>0, m>0\rbrace .\\
(Hints: travers an a, replace it by X, and move right to find the first c and replace it by Y.
then, traverse  left to find the second a. By this process, replace n number of c by Z.)\\

5. Design a turing machine to accepts the string L = \lbrace a^{m}b^{m+n}c^{n}, where m, n>0\rbrace .\\
6. Design a turing machine  by the transitional notation for the following languages\\
a) L =a^{n}b^{n}, n>0\\
b) L=\lbrace a^{n}b^{n}c^{m}d^{m}, where m,n \geq 1\rbrace \\
c) L=\lbrace a^{n}b^{n}c^{m}, where n,m \geq 1\rbrace \\

7. Make comments on the following statment that a finite state machine with two stacks is as powerful as a turing machine.\\

8. Design a two-stack PDA for adding and subtracting two numbers.\\
9.Design a two-stack PDA to accepts the string  L=\lbrace a^{m}b^{m+n}c^{n}, where m,n>0\rbrace .\\
\end{align}
\end{document}